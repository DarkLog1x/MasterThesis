\documentclass[12pt]{article}
\PassOptionsToPackage{hyphens}{url}\usepackage{hyperref}

\title{A Look at Implementing Security Rules and Safeguards in OpenStack for SNICScienceCloud}
\author{Aleksander Okonski \\ aleksander.oko@gmail.com}
\date{}

\begin{document}
\maketitle

\section{Background}
Uppsala University has several high performance computer nodes (HPCs). One of the uses for these HPCs is to run the SNICScienceCloud. This project lets researchers and professor's to run cloud based projects on the HPC recourses. The SNICScienceCloud uses OpenStack as its underlying cloud framework.


\section{Proposal}
OpenStack is an open source cloud computer solution that focuses on infrastructure as a service. OpenStack is used by the SNICScienceCloud as a platform to provide high performance computer nodes to university research and projects. This service is now expanding to the classroom where students are using SNIC cloud for cloud computing courses. As the SNIC cloud expands from research into education it provides larger security exposure. As machines are provisioned, and deployed into this hybrid cloud they will be accessible from the Internet. This will increase the risk of machines being compromised. The proposal for this project is to assess and implement a three step solution to monitor and protect the SNICScienceCloud infrastructure. The first part will be to write a user guild for setting up and provisioning vms. The second part will be to implement a "watch dog" VM that would monitor the other VMs in the project. The final part would be to look into implementing a system that would allow a VM to be monitored thought the underlying hypervisor.

\section{Tasks}

\begin{enumerate}
    \item Create a guide for uses to provision a VM.
    \item Create the "Watch Dog" VM.
    \item Ensure that the "Watch Dog" VM functions and is intuitive / easy to use.
    \item Study and look into a way that the hypervisor can be used to view VM activities.
\end{enumerate}

\section{Time Line}
\begin{enumerate}
    \item Week 1-2 Ramp up and write documentation on different specifications that are needed in the cloud.
    \item Week 3-6 Start building the "Watch Dog" server.
    \item Week 7 Finish up and test the "Watch Dog" server.
    \item Week 8-10 Start looking into ways of implementing a hypervisor based solution for viewing VM activities.
    \item Week 10-17 Implement the hypervisor based solution for viewing VM activities.
    \item Week 17-19 Fix bugs and ensure that all created components are working as expected.
    \item Week 20 Finish up the report and prepare for presentation.
\end{enumerate}

\begin{thebibliography}{1}

    \bibitem{nmap}
    https://nmap.org/

    \bibitem{OPsys}
    https://github.com/pyKun/openstack-systemtap-toolkit

    \bibitem{OPwiki}
    https://wiki.openstack.org/wiki/Successes

    \bibitem{OSip}
    https://openstack-in-production.blogspot.se/2015/09/ept-huge-pages-and-benchmarking.html

\end{thebibliography}
\end{document}

%There has been research conducted about the states of cloud security and Intrusion Detection Systems (IDS) systems on the cloud. There however has been little work done on the OpenStack IDS front and hypervisor level IDS\@. Bellow is a list of papers and web resources  that can be used for this research project.

%\subsection{Hypervisor IDS system:}
%Hypervisor-based Intrusion Detection

%\subsection{OpenStack IDS:}
%\url{https://www.openstack.org/summit/vancouver-2015/summit-videos/presentation/unobtrusive-intrusion-detection-in-openstack}

%\subsection{Cloud IDS:}
%Virtual Host based Intrusion Detection System for Cloud

%\subsection{System call tracing:}
%Nitro: Hardware-based System Call Tracing for Virtual Machines

%\subsection{Cloud Security:}
%An intrusion detection and prevention system in cloud computing: A systematic review

%\subsection{SystemTap on the OpenStack platform recourses:}
%\url{https://github.com/pyKun/openstack-systemtap-toolkit}
%\\
%\url{https://wiki.openstack.org/wiki/Successes}
%\\
%\url{https://openstack-in-production.blogspot.se/2015/09/ept-huge-pages-and-benchmarking.html}


