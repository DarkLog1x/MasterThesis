\documentclass[12pt]{article}
\PassOptionsToPackage{hyphens}{url}\usepackage{hyperref}

\title{A Look at an Intrusion Detection System for OpenStack in SNICScienceCloud}
\author{Aleksander Okonski \\ aleksander.oko@gmail.com}
\date{}

\begin{document}
\maketitle

\section{Background}
There has been research conducted about the states of cloud security and Intrusion Detection Systems (IDS) systems on the cloud. There however has been little work done on the OpenStack IDS front and hypervisor level IDS\@. Bellow is a list of papers and web resources  that can be used for this research project.

\subsection{Hypervisor IDS system:}
Hypervisor-based Intrusion Detection

\subsection{OpenStack IDS:}
\url{https://www.openstack.org/summit/vancouver-2015/summit-videos/presentation/unobtrusive-intrusion-detection-in-openstack}

\subsection{Cloud IDS:}
Virtual Host based Intrusion Detection System for Cloud

\subsection{System call tracing:}
Nitro: Hardware-based System Call Tracing for Virtual Machines

\subsection{Cloud Security:}
An intrusion detection and prevention system in cloud computing: A systematic review

\subsection{SystemTap on the OpenStack platform recourses:}
\url{https://github.com/pyKun/openstack-systemtap-toolkit}
\\
\url{https://wiki.openstack.org/wiki/Successes}
\\
\url{https://openstack-in-production.blogspot.se/2015/09/ept-huge-pages-and-benchmarking.html}

\section{Proposal}

OpenStack is an open source cloud computer solution that focuses on infrastructure as a service. OpenStack is used by the SNICScienceCloud as a platform to provide high performance computer nodes to university research and projects. This service is now expanding to the classroom where students are using SNIC cloud for cloud computing courses. As the SNIC cloud expands from research into education it provides larger security exposure. As machines are provisioned, and deployed into this hybrid cloud they will be accessible from the Internet. This will increase the risk of machines being compromised. The proposal for this project is to assess the feasibility and effectiveness of  intrusion detection system (IDS) for the OpenStack SNIC cloud. The goal would be to recognize that the configuration of a correctly provisioned machine for a particularly workload type. If the machine starts to differ from the desired configuration state then it should be detected, analyzed and action taken. The IDS solution would reside on the hypervisor level of OpenStack. There are several approaches for this solution.  We would consider systemTap on the OpenStack platform. SystemTap is a Linux kernel/system level probing tool. Another possible method that might work would be the use of Pthreads to extract similar information about the running VMs. This technique does not need to be localized to the SNICScienceCloud and can be expanded to fit any OpenStack platform and may even be implemented in other cloud solutions.
The outcome of this proof of concept will be the validation if this IDS approach is feasible, how effective it is, and how difficult to implement and operate at the cloud scale.

\section{Tasks}

\begin{enumerate}
    \item Look at different possible solutions into hypervisor IDS systems for OpenStack.
    \item Design and implement a working hypervisor IDS system for OpenStack.
    \item Test the created IDS system
    \item Document the created IDS system.
\end{enumerate}

\end{document}

