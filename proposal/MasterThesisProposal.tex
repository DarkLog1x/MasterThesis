\documentclass[12pt]{article}
\PassOptionsToPackage{hyphens}{url}\usepackage{hyperref}

\title{Implementing Security Rules, Safeguards, and IDS tools for Private Cloud Infrastructures}
\author{Aleksander Okonski \\ aleksander.oko@gmail.com}
\date{}

\begin{document}
\maketitle

\section{Background}
Cloud computing has revolutionized the way computational resources had been offered before. The philosophy of having dedicated in-house resources transformed into pay-as-you-go model for accessing large amount of computational and storage resources. Together with number of advantages, the technology also brings some serious challenges for the community. Security is one of the fundamental concerns both for the users and service providers.
\\
\\
UPPMAX is a local HPC centre at Uppsala University. Together with HPC setups, UPPMAX is also participating in national level cloud initiative called SNIC Science Cloud (SSC). SSC is an community cloud environment for Swedish academia, consists of three geographically distributed regions (HPC2N, C3SE and UPPMAX). The SNIC Science Cloud uses OpenStack as its underlying cloud framework. OpenStack is an open source cloud computing solution that focuses on Infrastructure-as-a-Service (IaaS). This service is now expanding to the classroom where students are using SNIC cloud for cloud computing courses. As the SNIC Science Cloud becomes more poplar among researchers and students the security surface gradually expands. As machines are provisioned, and deployed into this hybrid cloud they will be accessible from the Internet. This poses a security risk as these users may not be aware of security rules and proper guidelines to use when managing virtual machines (VMs) in the cloud. There have already been issues with several VMs being infected and joining bot-nets. It is therefor necessary to look into the ability to protect the SNIC Science Cloud and ensure users are following best practices.

\section{Proposal}
The proposal for this project is to assess and implement a three step solution to monitor and protect private cloud infrastructures based on Openstack solution. SSC will be the reference infrastructure. The first part will be to identify and write a set of recommendation for setting up and provisioning VMs. The second part will be to implement a ”watch dog” VM that would monitor the other VMs in the cluster externally for misconfiguration. This VM would be part of the deployed OpenStack cluster. The final part would be to look into implementing a system that would allow the internal processes of a VM to be monitored for internal configuration problems and threats.

\section{Tasks}

\begin{enumerate}
    \item Provide recommendations for secure VM provisioning on private cloud infrastructures.
    \item Create the ”Watch Dog” VM to aggregate logs and scan for misconfig- duration.
    \item Ensure that the ”Watch Dog” VM is properly tested and can be ex- tended.
    \item Study and implement a way to view VM activities and internal configurations.
\end{enumerate}

\section{Time Line}
\begin{enumerate}
    \item Week 1-2 Literature survey and write documentation on different spec- ifications that are needed in the cloud.
    \item Week 3-6 Start building the ”Watch Dog” server. Week 7 Finish up and test the ”Watch Dog” server.
    \item Week 7 Finish up and test the "Watch Dog" server.
    \item Week 8-10 Start looking into ways of implementing a hypervisor based solution for viewing VM activities and configuration.
    \item Week 10-17 Implement a method to view internal VM activities and configurations.
    \item Week 17-19 Fix bugs and ensure that all created components are working as expected.
    \item Week 20 Finish up the report and prepare for presentation.
\end{enumerate}

\begin{thebibliography}{1}

    \bibitem{nmap}
    https://nmap.org/

    \bibitem{OPsys}
    https://github.com/pyKun/openstack-systemtap-toolkit

    \bibitem{OPwiki}
    https://wiki.openstack.org/wiki/Successes

    \bibitem{OSip}
    https://openstack-in-production.blogspot.se/2015/09/ept-huge-pages-and-benchmarking.html

    \bibitem{OSIDS}
    https://www.openstack.org/summit/vancouver-2015/summit-videos/presentation/unobtrusive-intrusion-detection-in-openstack

    \bibitem{SIMS}
    https://en.wikipedia.org/wiki/Security\_information\_and\_event\_management
    \\
    https://www.splunk.com/en\_us/download.html

    \bibitem{SystemTap on the OpenStack platform recourses:}
    https://github.com/pyKun/openstack-systemtap-toolkit
    \\
    https://wiki.openstack.org/wiki/Successes
    \\
    https://openstack-in-production.blogspot.se/2015/09/ept-huge-pages-and-benchmarking.html


\end{thebibliography}
\end{document}

%There has been research conducted about the states of cloud security and Intrusion Detection Systems (IDS) systems on the cloud. There however has been little work done on the OpenStack IDS front and hypervisor level IDS\@. Bellow is a list of papers and web resources  that can be used for this research project.

%\subsection{Hypervisor IDS system:}
%Hypervisor-based Intrusion Detection

%\subsection{OpenStack IDS:}
%\url{https://www.openstack.org/summit/vancouver-2015/summit-videos/presentation/unobtrusive-intrusion-detection-in-openstack}

%\subsection{Cloud IDS:}
%Virtual Host based Intrusion Detection System for Cloud

%\subsection{System call tracing:}
%Nitro: Hardware-based System Call Tracing for Virtual Machines

%\subsection{Cloud Security:}
%An intrusion detection and prevention system in cloud computing: A systematic review

%\subsection{SystemTap on the OpenStack platform recourses:}
%\url{https://github.com/pyKun/openstack-systemtap-toolkit}
%\\
%\url{https://wiki.openstack.org/wiki/Successes}
%\\
%\url{https://openstack-in-production.blogspot.se/2015/09/ept-huge-pages-and-benchmarking.html}


