\documentclass[12pt]{article}
\usepackage{hyperref}
\usepackage{graphicx}
\graphicspath{{./}}
\usepackage{dirtytalk}
\usepackage{listings}
\lstset{basicstyle=\small\ttfamily,
    columns=flexible,
    breaklines=true
}


\title{Implementing Security Rules, Safeguards, and IDS tools for Private Cloud Infrastructures}
\author{Author: Aleksander Okonski -- aleksander.oko@gmail.com \\ Supervisor: Salman Toor \\ Review: Bjorn Victor }
\date{}


\begin{document}
\maketitle
\newpage
\tableofcontents
\newpage

\section{Background}
The cloud computing space has grown over the last several years. Business and Universities are looking at solutions to migrate their existing infrastructure to the cloud. There are several reasons for this type of business shift: costs, scalability, reliability \cite{DillonWuChang}. The cloud offers some precedented advantages to an standardized computational model. One is able to pay for only the resources used, with more recurses added/removed depending on the demand. Another advantage is the ability to spine up/destroy several machines with little overhead. Several companies are fronting the cloud revolution including Amazon, Google, Microsoft, and Digital Ocean.

\subsection{Cloud Models}
In the cloud computing space several different computational models exist \cite{wikipedia}.

\begin{itemize}
    \item Software as a Service (SaaS) allows for the user to utilize applications (I.E. Email, games, etc.) without the need to set up / worry about the underlying infrastructure.
    \item Platform as a Service (PaaS) give the user the ability to create applications (I.E. Web servers, databases, etc.) without the need to create the entire system from he ground up.
    \item Infrastructure as a Service (IaaS) gives the users a basic virtual machine with the user needing to setup all nectary functionality.
\end{itemize}

\subsection{Cloud Infrastructure}
At the most fundamental layer a cloud computer is a server running in a data-center that has a hypervisor which then contains and runs another virtual machine. These hypervisores are the backbone of cloud computing allowing several different


%talk about hypervisores

%Talk about how the cloud space looks like
%Talk about the diffrent models
%talk about openstack vs amazome / digitalocian
%Talk about security
%Talk about the security side of cloud and how its diffrent then the normal aspects
%

\section{Related Work}
A good starting article \cite{SurveyOfIDS}

\newpage
\bibliographystyle{plain}
\bibliography{thesis.bib}
\end{document}
